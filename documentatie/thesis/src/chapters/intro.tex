\chapter{Introduction}
\label{chapter:intro}
\abbrev{PNaCl}{Portable Native Client}
\abbrev{OpenGL}{Open Graphics Library}
\abbrev{WebGL}{Web Graphics Library}
\abbrev{JS}{Javascript}
\abbrev{SIMD}{Single Instruction Multiple Data}
\abbrev{API}{Application Programming Interface}
\abbrev{SDL}{Simple DirectMedia Layer}
\abbrev{GLUT}{OpenGL Utility Toolkit}
\abbrev{GLFW}{OpenGL Framework}
\abbrev{GDC}{Game Developers Conference}
\abbrev{LLVM}{Low Level Virtual Machine}
\abbrev{DOM}{Document Object Model}
\abbrev{Assimp}{Open Asset Import Library}
\abbrev{SVG}{Scalable Vector Graphics}
\abbrev{CSS}{Cascading Style Sheets}
\abbrev{GPU}{Graphics Processing Unit}
\abbrev{JIT}{Just In Time}
\abbrev{AOT}{Ahead Of Time}
\abbrev{VM}{Virtual Machine}
\abbrev{GUI}{Graphical User Interface}
\abbrev{HLSL}{High-level shading language}
\abbrev{WebRTC}{Web Real-Time Communication}
\abbrev{P2P}{Peer-to-peer}
\abbrev{GPGPU}{General-purpose computing on graphics processing units}
\abbrev{HUD}{Head-Up Display}

The scope of the paper is to implement and investigate the potential of implementing a
\project\ (Web Graphics Library) and web technologies and leveraging them in order to create an efficient game engine.

WebGL is a Khronos specification for the browsers which is modeled after OpenGL ES 2.0. Compared to OpenGL ES 2.0, it lacks some of the commands related to client side buffers as it is not practical to implement them in JavaScript. The specification also introduces Typed Arrays to JavaScript, as OpenGL needs to have typed data. On Windows, modern browsers implement WebGL using Angle. Angle is a wrapper for WebGL that translates all the WebGL calls to DirectX 9.0 and recompiles shaders to HLSL (High-level shading language). Because the use of the WebGL API can't be trusted (like in the case of desktop applications), Angle implements various workarounds for known driver bugs.


\section{Three.js: ups and downs}
\label{sec: ThreeJs}

One of the most popular WebGL libraries is ThreeJS. It’s one of the most used Javascript WebGL libraries on the web and supports multiple renderer backends with fallbacks if the web browser does not support WebGL, to Canvas2D, SVG (Scalable Vector Graphics) or CSS3D (Cascading Style Sheets). 
While it does more than what a renderer should do, by providing: Scene Management,
Resource Loader, Animation System and different camera types that are integrated in the
input system, it lacks a physics engine, a scripting engine, a user accessible input system and a lot more.
A number of libraries based on ThreeJS have appeared that try to fix these problems. For example, Physijs links ThreeJS to ammo.js, an Emscripten compiled Bullet Physics Engine.


\section{Emscripten versus Portable Native Client}
\label{sec: Emscripten, PNaCl}

In 2011, Mozilla introduced Emscripten, a C++ compiler that compiles C++ code to JavaScript, allowing the porting of existing C++ game engines to the web. They also created asm.js, a typed subset of JavaScript that can be compiled by the JavaScript interpreter directly to microprocessor instructions. They made it so that Emscripten generated JavaScript code can get the biggest performance benefit from this. The problem for Emscripten is that if you compile an engine designed for desktop use, you won't take benefit of the things that the browser offers (like input callbacks, image loading and decompressing, video and audio decompressing, playback capabilities and various other). To fix this problem, Emscripten has it's own implementation of popular libraries like SDL (Simple DirectMedia Layer), GLUT (OpenGL Utility Toolkit) or GLFW (OpenGL Framework) that use the browser's native system. Another problem of Emscripten is that JavaScript does not support threads; instead it uses workers. Workers are usually implemented by the interpreter as a separate thread which can communicate only over a message passing API (Application Programming Interface). This means that most of the algorithms implemented in modern engines that use multithreading will not be easily ported and also it introduces an overhead related to the memory copy necessary to transmit the message. The huge advantage is given by asm.js, which allows Emscripten compiled C++ code to achieve near native speed performance in your browser. Mozilla, together with Epic, announced that they ported Unreal Engine 4 to the browser, after last year porting the well known Unreal Engine 3 to the browser using Emscripten. Unity technologies also announced at GDC (Game Developers Conference) that their next Unity Engine will have the ability to export your game to the browser.


A competitive technology pushed by Google in the Chrome web browser is Portable Native Client (PNaCl) which allows C++ code to be compiled, but not in JavaScript. It compiles the code using a special compiler to LLVM (Low Level Virtual Machine) byte-code and then it's translated by the browser to native instructions on the client computer. The code is run in a special PNaCl sandbox and communicates with the browser using a message passing interface similar to that of a worker. It also has access to Pepper Plugin API which allows it to perform OpenGL ES 2.0 (Open Graphics Library) and various other actions. PNaCl has all the advantages of running native code, including threading, SIMD operations (Single Instruction Multiple Data) and memory management. By leveraging these, it is easier to port an existing game to PNaCl than it is to compile it with Emscripten. One of the downsides is that it doesn't support linking with dynamic libraries, so all the code needs to be statically linked during the linking phase of the compilation process. This may cause licensing issues with some libraries. Besides this, some of the new WebGL extensions take some time before appearing in the PNaCl API and you can't modify it in order to support them, because it's implemented in the browser. But, the major problem with PNaCl is the lack of adoption, it being currently available only on Chrome and some other major browsers (Firefox) have already announced that they are not interested in pursuing this path. There are currently only a couple of very successful games ported to PNaCl: one of them is an indie game, Bastion and the other one is From Dust, from Ubisoft.


Although these approaches have managed to fix the performance issue to some degree, they have done so by fighting the web technologies with custom APIs (Application Programming Interface) and custom implementations that work only on their browser and have not gained universal acceptance. This paper proposes to embrace what the web offers and take full advantage of it, by designing an engine around the APIs offered by a browser instead of trying to fit an existing game engine architecture in the browser.


\section{What problems are posed by a JavaScript implementation of a game engine ?}
\label{sec:game engine}

The main problem in designing an engine in the browser is that you only have one main loop from which you can call draw orders and that is a slow one. So all the extra computation necessary, like physics and game mechanics, need to be processed in external threads and then just sent to the main thread which runs the renderer and the input system. These are the only modules that need to be run from the main browser thread. They also need to communicate with other modules using web worker's message passing API.

Web workers have a message passing API that allows the programmer to send multiple JavaScript objects between them and the window. The objects are copied from one thread to another, which due to the dynamic nature of JavaScript objects takes a lot of time. In order to address this problem browsers have implemented Transferable Objects. Transferable Objects are objects which can be sent from one process to another without copying. The only transferable object is ArrayBuffer. When used, the ArrayBuffer objects content disappears from the senders object and appears in the receivers object, allowing for very fast memory transfers at about 6GB/s. (from reference \cite{bidelman11}) This allows texture decompression and scene loading to be done on the workers and then easily transferred to the main thread.



\section{Project Description}
\label{proj-desc}

The game engine's resource manager will be implemented in a separate web worker and will handle all the 3D Model Resources, Texture Resources, Material Resources, Font Resources, Skeleton Resources, Collision Resources, Physics Parameters, Game Map and many other objects that need loading in order for the scene to run. All the other workers will communicate with it through messages. An asset pipeline will be created in order for the Resource Manager to be as efficient as possible, so as not to waste scarce processing power on resource conversion.

The scene will be modeled after a tree structure, each node having its own property modifications that need to be applied on top of the ones in the parent. The nodes will also be able to have a script run after each frame or during a physics sync time. These scripts will be written in JavaScript in the form of a function, which receives as parameters: the game state, the scene tree and the time from the last callback. The function will have to return it's new node state and any modifications that need to be made to the game state. It will be tested if it is beneficial to run all the scripts in parallel, each in a separate web worker or sequentially in the same worker.


The game state is a dictionary, each script registers exclusive access to a number of keys and the other scripts only get a read-only copy of it. After each run, the scripts return the new values for the keys and the game state gets updated. This is one of the ways the game state can be implemented so that it offers non concurrent access to multiple threads that work with it.


The low level renderer will be residing on the main thread and will parse the main scene tree in order to draw the scene. Additional post processing needs to be done with the use of the GPU (Graphics Processing Unit). For example, the team at Mozilla published an article (from reference \cite{tian14}) in which they show how deferred shading can be done with little impact on performance using a new WebGL extension that requires only two shader passes and doing all the heavy lifting on the GPU. The scene format should be optimized for maximum efficiency. For example, objects that are static can be linked into a single mesh and only one draw call performed. A limit to this strategy is the fact that OpenGL ES 2.0 limits us to 16bit element array indexes, which in turn means that we must make multiple such objects in order to tackle this problem.

The input system is designed around the API's available to JavaScript in a browser. It will be able to use defined actions that can be triggered by a multitude or a combination of key presses or inputs from other devices, like mouse, gyroscope or controller. This will allow the game developer to use an abstraction around the key configurations.

Each scene object needs information that can be passed to the physics engine if necessary. The physics engine that will be used is the popular Bullet Physics Engine, which is compiled into JavaScript using Emscripten. There is a JavaScript wrapper made around it, named ammo.js that offers access to all its methods.
 


\subsection{Objective}
\label{sub-sec:proj-desc-objective}

In conclusion, the paper aims at trying to engineer a game engine using all the APIs available in the browser and to make the best use of them.



